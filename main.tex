% \iffalse
\let\negmedspace\undefined
\let\negthickspace\undefined
\documentclass[journal,12pt,twocolumn]{IEEEtran}
\usepackage{float}
\usepackage{circuitikz}
\usepackage{cite}
\usepackage{amsmath,amssymb,amsfonts,amsthm}
\usepackage{algorithmic}
\usepackage{graphicx}
\usepackage{textcomp}
\usepackage{xcolor}
\usepackage{txfonts}
\usepackage{listings}
\usepackage{amsmath}
\usepackage{enumitem}
\usepackage{mathtools}
\usepackage{gensymb}
\usepackage{comment}
\usepackage[breaklinks=true]{hyperref}
\usepackage{tkz-euclide} 
\usepackage{listings}
\usepackage{gvv}                                        
\def\inputGnumericTable{}                                 
\usepackage[latin1]{inputenc}                                
\usepackage{color}                                            
\usepackage{array}                                            
\usepackage{longtable}                                       
\usepackage{calc}                                             
\usepackage{multirow}                                         
\usepackage{hhline}                                           
\usepackage{ifthen}                                           
\usepackage{lscape}
\newtheorem{theorem}{Theorem}[section]
\newtheorem{problem}{Problem}
\newtheorem{proposition}{Proposition}[section]
\newtheorem{lemma}{Lemma}[section]
\newtheorem{corollary}[theorem]{Corollary}
\newtheorem{example}{Example}[section]
\newtheorem{definition}[problem]{Definition}
\newcommand{\BEQA}{\begin{eqnarray}}
\newcommand{\EEQA}{\end{eqnarray}}
\newcommand{\define}{\stackrel{\triangle}{=}}
\theoremstyle{remark}
\newtheorem{rem}{Remark}
\begin{document}
\bibliographystyle{IEEEtran}
\vspace{3cm}
\title{Filter Design}
\author{EE23BTECH11041 - Md Ayaan Ashraf}
\maketitle
\newpage
\bigskip
\renewcommand{\thefigure}{\arabic{figure}}
\bibliographystyle{IEEEtran}
\section{\textbf{Introduction}}
We are supposed to design the equivalent FIR and IIR filter realizations for given filter number.  
This is a bandpass filter whose specifications are available below.
\section{\textbf{Filter Specifications}}
\subsection{The Digital Filter}
\begin{enumerate}
\item {Passband:}
The passband is from \\
\{4 + 0.6(j)\}kHz to \{4 + 0.6(j+2)\}kHz. \\
where 
\begin{align}
    j=\brak{r-11000} \mod \sigma
\end{align}
where $\sigma$ is sum of digits of roll number and $r$ is roll number.\\
\begin{align}
    r&=11041\\
    \sigma  &=7\\
    j&=6
\end{align}
 substituting $j =6$ gives the passband
range for our bandpass filter as $7.6$ kHz - $8.8$ kHz.  Hence, the un-normalized discrete time filter
passband frequencies are $F_{p1} = 7.6$ kHz
and $F_{p2} = 8.8$ kHz. \\
The corresponding normalized digital filter passband frequencies are
\begin{align}
    \omega_{p1} &= 2\pi\frac{F_{p1}}{F_s} = 0.32 \pi\\
    \omega_{p2} &= 2\pi\frac{F_{p2}}{F_s}  =0.37 \pi
\end{align}

\item {Tolerances:}  The passband ($\delta_1$) and stopband ($\delta_2$) tolerances are given to
be equal, so we let $\delta_1 = \delta_2 = \delta = 0.15$.
\\
\item { Stopband:}  The {transition band} for bandpass filters is $\Delta F = 0.3$ kHz on either side of the passband.

\begin{align}
    F_{s1} &= 7.6-0.3 = 7.3 \text{kHz}\\
    F	_{s2} &= 8.8+0.3 = 9.1  \text{kHz}
\end{align}
\begin{align}
    \omega_{s1} &= 2\pi\frac{F_{s1}}{F_s} = 0.304 \pi\\
     \omega_{s2} &= 2\pi\frac{F_{s2}}{F_s} = 0.379 \pi\\
\end{align}
\end{enumerate}
\subsection{The Analog filter}
In the bilinear transform, the analog filter frequency ($\Omega$) is related to the corresponding digital filter frequency\brak{\omega} :
\begin{align}
  \Omega = \tan \frac{\omega}{2}  
\end{align}
Using this relation, we obtain the analog passband and stopband frequencies as:
$\Omega_{p1} = 0.5497$, $\Omega_{p2} = 0.6568$ and $\Omega_{s1} = 0.5174$, $\Omega_{s2} = 0.6773$
respectively.

\section{\textbf{The IIR Filter Design}}
We are supposed to design filters whose stopband is monotonic and passband equiripple.  
Hence, we use the Chebyschev approximation to design our bandpass IIR filter.
\subsection{\textbf{The Analog Filter}}
\begin{enumerate}

\item \underline{Low Pass Filter Specifications:}  Let $H_{a, BP}(j\Omega)$ be the desired analog bandpass filter,  with the specifications provided in Section 2.2, and $H_{a,LP}(j\Omega_L)$ be the equivalent low pass filter, then
\begin{equation}
\Omega_L = \frac{\Omega^2 - \Omega_0^2}{B\Omega} \label{eq:freq_transform}
\end{equation}
where $\Omega_0 = \sqrt{\Omega_{p1}\Omega_{p2}} = 0.6008$ and $B = \Omega_{p2} - \Omega_{p1} = 0.1071$.\\
Substituting $\Omega_{s1}$ and $\Omega_{s2}$ in \eqref{eq:freq_transform} we obtain the stopband edges of lowpass filter 
\begin{align}
    \Omega_{Ls1} &= \frac{\Omega_{s1}^2 - \Omega_0^2}{B\Omega_{s1}} = -1.682\\
    \Omega_{Ls2} &= \frac{\Omega_{s2}^2 - \Omega_0^2}{B\Omega_{s2}} = 1.347
\end{align}
And we choose the minimum of these two stopband edges
\begin{align}
    \Omega_{Ls} = \mbox{min}(\vert \Omega_{Ls_1}\vert,\vert \Omega_{Ls_2}\vert) = 1.347.
\end{align}
\item \underline{The Low Pass Chebyschev Filter Parameters:} The magnitude of frequency response of the low pass filter is given by 
\begin{align}
    \vert H_{a,LP}(j\Omega_L)\vert^2 = \frac{1}{1 + \epsilon^2c_N^2(\Omega_L/\Omega_{Lp})} \label{eq:mag_freq_response}
\end{align}
The passband edge of the low pass filter is chosen as $\Omega_{Lp}=1$.
Therefore ,
\begin{align}
    \vert H_{a,LP}(j\Omega_L)\vert^2 = \frac{1}{1 + \epsilon^2c_N^2(\Omega_L)} \label{eq:specification}
\end{align}
Here $c_N$ denote the chebyshev polynomials for a particular order $N$ of the filter.
\begin{align}
    c_N(x) &= \cosh(N \cosh^{-1}x) , x=\Omega_{L}\\
    c_0(x) &= 1 \\
    c_1(x) &= x
\end{align}
There exists a recurssive relation from which all the polynomials can be found out.
\begin{align}
    c_{N+2} &= 2xc_{N+1} - c_{N}  \label{eq:cheby_poly_relation}
\end{align}
Imposing the band restrictions on \eqref{eq:mag_freq_response} \\
\begin{align}
    \vert H_{a,LP}(j\Omega_L)\vert^2 < \delta_{2} \hspace{5pt} \text{for}\hspace{5pt} \Omega_L = \Omega_{Ls}\\
    1-\delta_{1}<\vert H_{a,LP}(j\Omega_L)\vert^2 < 1 \hspace{5pt} \text{for}\hspace{5pt} \Omega_L = \Omega_{Lp}
\end{align}
we obtain :
\begin{eqnarray}
\label{lpdesign}
\frac{\sqrt{D_2}}{c_N(\Omega_{Ls})} \leq \epsilon \leq \sqrt{D_1}, \nonumber \\
N \geq \left\lceil \frac{\cosh^{-1}\sqrt{D_2/D_1}}{\cosh^{-1}\Omega_{Ls}} \right\rceil,
\end{eqnarray}
where $D_1 = \frac{1}{(1 - \delta)^2}-1$ and $D_2 = \frac{1}{\delta^2} - 1$ and $\left \lceil . \right \rceil$ is known as the ceiling operator . 
\input{table/table1}
we get $N\geq 4$ and $0.3268 \leq \epsilon \leq 0.61$\\
\item \underline{The Low Pass Chebyschev Filter:}  Thus, we obtain
\begin{equation}
\label{lpsqfinal}
\vert H_{a,LP}(j\Omega_L)\vert^2 = \frac{1}{1 + 0.16c_4^2(\Omega_L)}
\end{equation}
where
\begin{equation}
c_4(x) = 8x^4 - 8x^2 + 1.	
\end{equation}
The poles of the frequency response in (\ref{eq:mag_freq_response}) lying in the left half plane are in general obtained as 
\begin{align}
    p(k) = -\sinh{\phi}\sin{\phi(k)}+j\cosh{\phi}\cos{\phi(k)}
\end{align}
where
\begin{align}
    \phi &= \frac{1}{N}\sinh^{-1}{\brak{\frac{1}{\epsilon}}} \\
    \phi(k) &= \frac{(2k+1)}{N}\pi \hspace{5pt} k = 0, \dots, N-1
\end{align}

The following code generates the poles for $N=4$, stores it in a .txt file and plots the pole-zero plot in Figure 1,
\begin{lstlisting}
    https://github.com/Ashraf-1508/Filter-Design/blob/main/codes/pole_zero.py
\end{lstlisting}
\begin{figure}[h!]
    \centering
    \includegraphics[width=\columnwidth]{figs/pole-zero.png}
    \caption{pole-zero plot}
    \label{fig:p0}
\end{figure}
And the poles are stored into the following .txt file,
\begin{lstlisting}
    https://github.com/Ashraf-1508/Filter-Design/blob/main/codes/poles.txt
\end{lstlisting}

Thus, for N even, the low-pass stable Chebyschev filter, with a gain $G$ has the form (Only the poles on the left side of the $j\omega$ axis would be considered to ensure stability of the filter)
{\tiny{\begin{equation}
\label{poleleft}
H_{a,LP}(s_L) = \frac{G_{LP}}{(s_L-p(1))(s_L-p(2))(s_L-p(3))(s_L-p(4))}
\end{equation}}}
Substituting $N = 4$, $\epsilon = 0.4$ and $H_{a,LP}(j) = \frac{1}{\sqrt{1+\epsilon^2}}$, we obtain 
{\tiny{\begin{equation}
\label{lpfinal}
H_{a,LP}(s_L) = \frac{0.3125}{s_L^4 + 1.1068s_L^3 + 1.6125s_L^2+0.9140s_L + 0.3366}
\end{equation}}}

In Figure 2 we plot $|H(j\Omega)|$ using (\ref{lpsqfinal}) and (\ref{lpfinal}), thereby verifying that our low-pass Chebyschev filter design meets the specifications.
\begin{figure}
\label{fig1}
\includegraphics[width = 10cm]{figs/figure1.png}
\caption{The Analog Low-Pass Frequency Response for $0.35 \leq \epsilon \leq 0.6$}
\end{figure}
\newpage
\item \underline{The Band Pass Chebyschev Filter:} The analog bandpass filter is obtained from (\ref{lpfinal}) by substituting
$s_L = \frac{s^2 + \Omega_0^2}{Bs}$.  Hence
\begin{equation}
H_{a,BP}(s) = G_{BP}H_{a,LP}(s_L)\vert_{s_L = \frac{s^2 + \Omega_0^2}{Bs}},
\end{equation}
where $G_{BP}$ is the gain of the bandpass filter.  After appropriate substitutions, and evaluating the gain 
such that $H_{a,BP}(j\Omega_{p1}) = 1$, we obtain
{
\tiny
\begin{equation}
\label{bpfinal}
H_{a,BP}(s) = \frac{4.3489\times 10^{-5}s^4}{s^8+0.1179s^7+1.4320s^6+0.1262s^5+0.7625s^4+0.0446s^3+0.1789s^2+0.0052s+0.0156}
\end{equation}
}
Where,
\begin{align}
    G_{BP} = 1.0788
\end{align}
The above substitution is done by the following code,
\begin{lstlisting}
    https://github.com/Ashraf-1508/Filter-Design/blob/main/codes/coeff_analog.py
\end{lstlisting}
And the coefficients are stored into the .txt file,
\begin{lstlisting}
    https://github.com/Ashraf-1508/Filter-Design/blob/main/codes/coefficients_analog.txt
\end{lstlisting}
In Figure 4, we plot $\vert H_{a,BP}(j\Omega)\vert$ as a function of $\Omega$ for both positve as
well as negative frequencies.  We find that the passband and stopband frequencies in the figure
match well with those obtained analytically through the bilinear transformation.
\end{enumerate}


\begin{figure}
\label{fig2}
\includegraphics[width = 10cm]{figs/figure2.png}
\caption{The magnitude response plots from the specifications in Equation \ref{lpsqfinal} and the design in Equation \ref{lpfinal}}
\end{figure}

\begin{figure}
\label{fig3}
\includegraphics[width = 10cm]{figs/figure3.png}
\caption{The analog bandpass magnitude response plot from Equation \ref{bpfinal}} 
\end{figure}


\subsection{\textbf{The Digital Filter}}
From the bilinear transformation, we obtain the digital bandpass filter from the corresponding analog filter as
\begin{eqnarray}
\label{analdig}
H_{d,BP}(z) = GH_{a,BP}(s)\vert_{s = \frac{1-z^{-1}}{1 + z^{-1}}}
\end{eqnarray}
where $G$ is the gain of the digital filter.  From (\ref{bpfinal}) and (\ref{analdig}), we obtain

\begin{eqnarray}
H_{d,BP}(z) = G \frac{N(z)}{D(z)}
\end{eqnarray}
where $G = 4.3489 \times 10^{-5}$,
\begin{eqnarray}
N(z)=  1 - 4 z^{-2} + 6 z^{-4} - 4z^{-6} + z^{-8} 
\end{eqnarray}
and
{\tiny{\begin{eqnarray}
D(z) = 3.6830  -13.7277z^{-1} + 33.2138z^{-2}  -51.2028z^{-3}+  59.5578z^{-4}\nonumber \\
  -49.0243z^{-5}+   30.4476z^{-6}  -12.0480z^{-7} +   3.0950z^{-8}
\end{eqnarray}}}\\\\
The substitution is done by the code,
\begin{lstlisting}
    https://github.com/Ashraf-1508/Filter-Design/blob/main/codes/coeff_digital.py
\end{lstlisting}
And the the coefficients are then stored in this .txt file,
\begin{lstlisting}
    https://github.com/Ashraf-1508/Filter-Design/blob/main/codes/coefficients_digital.txt
\end{lstlisting}
The plot of $|H_{d,BP}(z)|$ with respect to the normalized angular freqency (normalizing factor $\pi$) is available in Figure 5.  Again we
find that the passband and stopband frequencies meet the specifications well enough.
\begin{figure}
\label{fig4}
\includegraphics[width = 10cm]{figs/figure4.png}
\caption{The magnitude response of the bandpass digital filter designed to meet the given specifications}
\end{figure}    

\section{\textbf{The FIR Filter}}
We design the FIR filter by first obtaining the (non-causal) lowpass equivalent using the Kaiser window
and then
converting it to a causal bandpass filter.

\subsection{\textbf{The Equivalent Lowpass Filter}}
The lowpass filter has a passband frequency $\omega_l$ and transition band $\Delta \omega = 2\pi \frac{\Delta F}{F_s} = 0.0125\pi$.
The stopband tolerance is $\delta$.
\begin{enumerate}
\item  The {\em passband frequency $\omega_l$}  is defined as $\omega_l = \frac{\omega_{p1} - \omega_{p2}}{2}$.  Substituting the values of $\omega_{p1}$ and $\omega_{p2}$ from section 2.1, we obtain $\omega_l = 0.025\pi$.

\item {\em The impulse response $h_{lp}(n)$} of the desired lowpass filter with cutoff frequency $\omega_l$
is given by
\begin{eqnarray}
\label{firlpdef}
h_l(n) = \frac{\sin(n\omega_l)}{n\pi}w(n),
\end{eqnarray}
where $w(n)$ is the Kaiser window obtained from the design specifications.
\end{enumerate}
\subsection{\textbf{The Kaiser Window}}
The Kaiser window is defined as
\begin{eqnarray}
\label{kaiser}
w(n) &=& \frac{I_0\left[ \beta N \sqrt{1 - \left(\frac{n}{N}\right)^2} \right]}{I_0(\beta N)},
\indent -N \leq n \leq N, \indent \beta > 0 \nonumber \\
&=& 0 \hspace{5cm} \mbox{otherwise,}
\end{eqnarray}
where $I_0(x)$ is the modified Bessel function of the first kind of order zero in $x$ and $\beta$
and $N$ are the window shaping factors.  In the following,
we find $\beta$ and $N$ using the design parameters in section 2.1.

\begin{enumerate}
\item  N is chosen according to
\begin{equation}
N \geq \frac{A-8}{4.57\Delta \omega},
\end{equation}
where $A = -20\log_{10}\delta$.  Substituting the appropriate values from the design specifications, we obtain
$A = 16.4782$ and $N \geq 48$.
\\
\item  $\beta$ is chosen according to
{\tiny\begin{eqnarray}
\label{kaisercond}
\beta N = \left\{ \begin{array}{ll} 0.1102(A-8.7) & A > 50 \\
0.5849(A-21)^{0.4}+ 0.07886(A-21) & 21 \leq A \leq 50 \\
0 & A < 21\end{array} \right.
\end{eqnarray}}
In our design, we have $A = 16.4782 < 21$.  Hence, from (\ref{kaisercond}) we obtain $\beta = 0$.  
\\
\item We choose $N = 100$, to ensure the desired low pass filter response.  Substituting in (\ref{kaiser})
gives us the rectangular window
\begin{eqnarray}
\label{rect}
w(n) &=& 1, \indent -100 \leq n \leq 100 \nonumber \\
&=& 0 \hspace{6mm} \mbox{otherwise}
\end{eqnarray}
\end{enumerate}

From (\ref{firlpdef}) and (\ref{rect}), we obtain the desired lowpass filter impulse response
\begin{eqnarray}
\label{firlpfinal}
h_{lp}(n) &=& \frac{\sin(\frac{n\pi}{40})}{n\pi} \indent -100 \leq n \leq 100 \nonumber \\
&=& 0, \hspace{2cm} \mbox{otherwise}
\end{eqnarray}
The response of the filter in (\ref{firlpfinal}) is shown in Figure 6.

\begin{figure}
\label{fig5}
\includegraphics[width = 10cm]{figs/figure5.png}
\caption{The frequency and the impulse response of the FIR lowpass digital filter designed to meet the given specifications} 
\end{figure}

\subsection{\textbf{The FIR Bandpass Filter}}
The centre of the passband of the desired bandpass filter was found to be $\omega_c = 0.275\pi$ in Section
2.1.  The impulse response of the desired bandpass filter is obtained from the impulse response of the
corresponding lowpass filter as
\begin{eqnarray}
h_{bp}(n) = 2h_{lp}(n)cos(n\omega_c)
\end{eqnarray}
Thus, from (\ref{firlpfinal}), we obtain
\begin{eqnarray}
\label{firbpfinal}
h_{bp}(n) &=& \frac{2\sin(\frac{n\pi}{40}) \cos(0.342n\pi)}{n\pi} \indent -100 \leq n \leq 100 \nonumber \\
&=& 0, \hspace{4cm} \mbox{otherwise}
\end{eqnarray}
%
The frequency response of the FIR bandpass filter designed to meet the given specifications is plotted in Figure 7.
\begin{figure}[h!]
\label{fig6}
\includegraphics[width = 10cm]{figs/figure6.png}
\caption{The frequency response of the FIR bandpass digital filter designed to meet the given specifications} 
\end{figure}

\end{document}
